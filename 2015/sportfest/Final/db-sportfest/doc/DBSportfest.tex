\documentclass[pagesize,11pt,twoside]{scrartcl}

\usepackage{geometry}
\usepackage{layout}

\usepackage[latin1,utf8x]{inputenc}

\usepackage[ngerman]{babel}
\usepackage[intlimits]{amsmath}
\usepackage{amsfonts}
\usepackage{amssymb}
\usepackage{array}
\usepackage{pst-all}
\usepackage{pstricks-add}
\usepackage{graphicx}

\usepackage{ulem}%Durchstreichen mit \xout{} oder \sout{}
\usepackage{scrpage2}%Kopf und Fu�zeilen

%\usepackage{listings}

\pagestyle{scrheadings}
%\chead{\automark[section]{section}}
\chead{Datenbank Miniwelt Sportfest}
\ohead{FU-Berlin\\LWB Informatik}

\ihead{H. Huth, K. Petri,\\P. Kreißig}
%\setheadtopline{1pt}
\setheadsepline{0.4pt}
\ofoot{SS 2014/15}
\setfootsepline{0.4pt}
\ifoot{\pagemark}
\cfoot{}

\parindent=0cm
\parskip=1.5ex

\geometry{includehead=true,includefoot=true,height=28.6cm,width=16.5cm,headheight=2.5cm,
  footskip=10mm,headsep=3mm,inner=2cm}

\definecolor{dschungel}{cmyk}{0.99,0,0.52,0}
\definecolor{Hinten}{cmyk}{0,0,0,0.2}

\DeclareMathAlphabet{\mathpzc}{OT1}{pzc}{m}{it}
\def\FZ#1#2{#1_{\mbox{\footnotesize #2}}}
\def\E#1{\,\mathrm{#1}}
\newcommand*\diff{\mathop{}\!\mathrm{d}}

\def\Lsg#1{{\color{blue}#1}}
\def\OL#1{\overline{#1}}
\def\rsa{$\rightsquigarrow$}

\definecolor{CEntity}{cmyk}{0,0.32,0.52,0}
\definecolor{CRelation}{cmyk}{0.26,0,0.76,0}
\definecolor{CAttribut}{cmyk}{0,0,1,0}
\def\Entity#1{\rnode{#1}{\psframebox[fillcolor=CEntity,fillstyle=solid]{#1}}}
\def\Relation#1#2#3#4#5#6{\dianode[fillcolor=CRelation,fillstyle=solid]{#1}{\parbox{1,7cm}{\centering #2}}\ncline{#1}{#3}\ncput*{#4}
	\ncline{#1}{#5}\ncput*{#6}}
\def\Attribut#1#2{\ovalnode[fillcolor=CAttribut,fillstyle=solid]{#1}{#1}\ncline{#1}{#2}}
\def\Key#1#2{\ovalnode[fillcolor=CAttribut,fillstyle=solid]{#1}{\underline{#1}}\ncline{#1}{#2}}
\def\Trafo#1#2#3{\fbox{\parbox[t]{#1}{\par{\centering\textbf{#2}\par} #3}}} % \par, weil centering nur im Absatz funktioniert
\def\TabKopf#1#2{\multicolumn{#2}{l}{\rule{1em}{0ex}\rule{0cm}{2.73ex}\textbf{#1}}\\\hline}

\begin{document}
%\layout
%\newpage
\section{Allgemeines}
Unsere Arbeitsgruppe hat sich für das Modell eines Sportfestes entschieden. In unserer Lehrtätigkeit
ist zu sehen, dass sich die Sportfeste in der Vergangenheit in ihrer Durchführung stark geändert haben.
Noch vor einiger Zeit mussten alle Schüler aller Klassen an allen vorgeschriebenen Disziplinen teilnehmen.
Derzeit erkennen wir einen Trend, dass Schüler die Disziplinen
nach ihren eigenen Interessen wählen können. 
Einem solchen Trend folgt unsere Miniwelt eines Schulsporfestes.

\section{Beschreibung der Miniwelt Sportfest in Textform}
Für ein Schulsportfest der Klassenstufen 5-13 soll eine Datenbank modelliert werden.

Die Wettkämpfe finden in Doppeljahrgangsstufen (5/6, 7/8, 9/10, 11/12) statt.
Schüler der Klasse 13 und Lehrer stellen Kampfrichter und Helfer für die einzelnen Sportarten.
Jede Klasse hat einen Lehrer als Betreuer.
Das Sportfest findet auf verschiedenen Sportanlagen statt.
Zu einer Anlage gehören verschiedenen Materialien in unterschiedlichen Anzahlen.
Auf einer Anlage können mehrere Wettkämpfe stattfinden, jedoch nicht zur gleichen Zeit.
An jeder Anlage wird verschieden lange Sport getrieben.
Jeder Schüler nimmt an wenigstens 3 und maximal 5 Wettkämpfen teil.
Es gibt verschiedene Sportarten.
In jeder Sportart finden Wettkämpfe in verschiedenen Wettkampfklassen getrennt nach 
Doppeljahrgangsstufe und Geschlecht statt.
Die Klassenstufe eines Schülers muss zur Wettkampfklasse passen.
Jeder Schüler wird durch Schülernummer, Name und Vorname charakterisiert.
Jeder Lehrer wird durch Lehrernummer, Name und Vorname charakterisiert.
Von vorhandenen Materialien werden Nummer, Bezeichnung und Anzahl erfasst.
Eine Sportanlage ist durch ihre Nummer und ihre Bezeichnung charakterisiert.

\subsection{Didaktische Reduzierung}
Der Datenumfang eines realen Sportfestes überschreitet die Übersichtlichkeit.
Wir werden auf die Doppeljahrgangsstufe 5/6 verzichten und auch die Anzahl der Schüler, 
Wettkämpfe, Wettkampfteilnahmen und Lehrer
gering halten.

Dadurch passen alle Daten auf eine DIN A4 Seite und Abfrageergebnisse lassen sich schnell 
auf ihre Korrektheit kontrollieren.
Dadurch lassen sich einige Bedingungen der Miniwelt nicht mehr aufrecht erhalten.
Z.B. nehmen nicht mehr alle Schüler an 3 bis 5 Wettkämpfen teil.
Die Entität Klasse blieb als Hinweis in der Populationsangabe. Eigentlich müsste jede Klasse
einen Klassenschlüssel erhalten, dr dann auch beim Schüler steht. Im nächsten Schuljahr würde sich nur der
Klassenname ändern und der Schüler wäre automatisch versetzt. Hier könnt die Datenbank noch weiter 
ausgebaut werden. Dann kämme auch die Beziehung gehtIn wieder hinzu.


%\end{enumerate}

\newpage
\section{ER-Modell Sportfest}
Darstellung unserer Miniwelt im ER-Modell inklusive vollständiger Attributierung, Schlüsselangaben und 
Komplexitäten.

\psset{unit=0.8cm}
\begin{pspicture}(-10,-10)(10,10)
%\rput(5,-6){\Entity{Sportart} %
%	\rput(2,-0.6){\Attribut{SPName}{Sportart}}
%	\rput(1.5,0.6){\Key{SpID}{Sportart}}
%	}
\rput(-3.5,7.5){\Entity{Sportanlage} %
	\rput(-5,-0.6){\Key{AnlagenID}{Sportanlage}}
	\rput(-5.5,0.6){\Attribut{AnlagenName}{Sportanlage}}
	}
%\rput(7,-2){\Relation{istAuf}{ist auf}{Sportart}{(1:3)}{Sportanlage}{(0:$\infty$)}}
\rput(5.5,8){\Entity{Material}
	\rput(-1.5,1.5){\Attribut{MatAnzahl}{Material}}
	\rput(1.5,0.8){\Attribut{MatName}{Material}}
	\rput(1.5,-0.4){\Key{MatID}{Material}}
	}
\rput(2,6){\Relation{benoetigt}{benötigt}{Sportanlage}{(1:$\infty$)}{Material}{(0:$\infty$)}
	\rput(1,-0.5){\Attribut{Anzahl}{benoetigt}}
	}
\rput(1,0){\Entity{Wettkampf}
	\rput(2.5,0){\Attribut{Geschlecht}{Wettkampf}}
	\rput(2,1){\Attribut{WettKlasse}{Wettkampf}}
	\rput(-5,0){\Attribut{WettName}{Wettkampf}}
	\rput(-4.5,1){\Key{WettID}{Wettkampf}}
	\rput(1,2){\Attribut{Termin}{Wettkampf}}
	}
\rput(-3,4){\Relation{findetStatt}{findet statt}{Wettkampf}{(0:3)}{Sportanlage}{(1:$\infty$)}
	\rput(-6,0){\Attribut{Dauer}{findetStatt}}
	}
%\rput(1,-1){\Relation{gehoert}{geh�rt zu}{Wettkampf}{(1:1)}{Sportart}{(1:$\infty$)}}
\rput(-4,-8){\Entity{Schueler}
	\rput(-4,2.5){\Attribut{SchGebDatum}{Schueler}}
	\rput(-4.3,0.5){\Attribut{SchName}{Schueler}}
	\rput(-4.5,-0.5){\Attribut{SchVorname}{Schueler}}
	\rput(-4.3,1.5){\Key{SchID}{Schueler}}
	\rput(-4,-1.5){\Attribut{SchGeschlecht}{Schueler}}
	}
\rput(-4.5,-4){\Relation{nimmtTeil}{nimmt teil}{Wettkampf}{(0:$\infty$)}{Schueler}{(3:5)}
	\rput(-5,1){\Attribut{Platz}{nimmtTeil}}
	\rput(-5,2){\Attribut{Ergebnis}{nimmtTeil}}
	}

\rput(6,-8){\Entity{Lehrer}
	\rput(2,1){\Attribut{LName}{Lehrer}}
	\rput(2,0){\Attribut{LVorname}{Lehrer}}
	\rput(2,-1){\Key{LID}{Lehrer}}
	}
\rput(1,-14){\Entity{Klasse}
	\rput(-2.5,-1){\Key{KlName}{Klasse}}
	}
\rput(5,-11){\Relation{betreut}{betreut}{Lehrer}{(0:1)}{Klasse}{(1:$\infty$)}}

\rput(7,-4){\Relation{verantw1}{verantw1}{Lehrer}{(0:1)}{Wettkampf}{(1:$\infty$)}
	\rput(-1.5,1.3){\Attribut{Funktion}{verantw1}}
	}
\rput(-3,-11){\Relation{gehtIn}{geht in}{Schueler}{(1:1)}{Klasse}{(1:$\infty$)}}

\rput(1,-5){\Relation{verantw2}{verantw2}{Schueler}{(0:1)}{Wettkampf}{(1:$\infty$)}
	\rput(-1,-1){\Attribut{Funktion}{verantw2}}
	}
\end{pspicture} 

\newpage
\def\SchuelerTraf{ %
	\Entity{Schueler}
				\rput(-4,1.5){\Key{SchID}{Schueler}}
					\rput(-3.8,-0.5){\Attribut{SchVorname}{Schueler}}
					\rput(-2,-1.5){\Attribut{SchGebDatum}{Schueler}}
					\rput(-4,0.5){\Attribut{SchName}{Schueler}}
					\rput(-0.5,1.5){\Attribut{SchGeschlecht}{Schueler}}
	}
\def\WettkampfTraf{
	\Entity{Wettkampf}
			\rput(-0.2,1.1){\Attribut{Geschlecht}{Wettkampf}}
			\rput(1.2,-0.8){\Attribut{WettKlasse}{Wettkampf}}
			\rput(-0.3,-1.7){\Attribut{WettName}{Wettkampf}}
			\rput(1.5,0.2){\Key{WettID}{Wettkampf}}
			\rput(-3,-1.2){\Attribut{Termin}{Wettkampf}}
	}
\section{Transformation des ER-Modells in eine Menge von Relationen}
\vspace{-0.3cm}
Für die Transformation geben wir als Teilüberschriften die verwendeten Transformationsregeln an.
 Unter den schematischen Darstellungen erscheinen die resultierenden Relationen.
\subsection{Transformation von 1:n Beziehungstypen}
\begin{pspicture}(0,-2.5)(0,0)
\rput(10,0){ %
	\rput(-6,0){\Entity{Lehrer}
		\rput(-2,-1){\Attribut{LName}{Lehrer}}
		\rput(0,-2){\Attribut{LVorname}{Lehrer}}
		\rput(2,-1){\Key{LID}{Lehrer}}
		}
	\rput(6,0){\Entity{Klasse}
		\rput(-1,-1){\Key{KlName}{Klasse}}
		}
	\rput(0,0){\Relation{betreut}{betreut}{Lehrer}{(0:1)}{Klasse}{(1:$\infty$)}}
}
\end{pspicture}

\Trafo{6cm}{Lehrer}{\underline{LID}, LName, LVorname, {\color{red}KlName}}\hfill
\Trafo{3cm}{Klasse}{\underline{KlName}}

\begin{pspicture}(0,-2.5)(0,1)
\rput(9,0){ %
	\rput(-6,0){\Entity{Lehrer}
			\rput(-2,-1){\Attribut{LName}{Lehrer}}
			\rput(0,-2){\Attribut{LVorname}{Lehrer}}
			\rput(2,-1){\Key{LID}{Lehrer}}
			}
	\rput(6,0){\WettkampfTraf}
	\rput(0,0){\Relation{verantw1}{verantw1}{Lehrer}{(0:1)}{Wettkampf}{(1:$\infty$)}
		\rput(-2.5,-1){\Attribut{Funktion}{verantw1}}
		}
}
\end{pspicture}

\Trafo{6cm}{Lehrer}{\underline{LID}, LName, LVorname, {\color{red}WettID, Funktion}}\hfill
\Trafo{6cm}{Wettkampf}{\underline{WettID},
WettKlasse, WettName, Geschlecht, Termin}

\begin{pspicture}(0,-2.5)(0,2.5)
\rput(10.5,0){ %
	\rput(-6,0){\SchuelerTraf}
	\rput(6,0){\WettkampfTraf}
	\rput(0,0){\Relation{verantw2}{verantw2}{Schueler}{(0:1)}{Wettkampf}{(1:$\infty$)}
		\rput(-2.5,-1.5){\Attribut{Funktion}{verantw2}}
		}
}
\end{pspicture}

\Trafo{7.5cm}{Schüler}{\underline{SchID}, SchName, SchVorname, SchGeschlecht, SchGebDatum, {\color{red}WettID, Funktion}}\hfill
\Trafo{6cm}{Wettkampf}{\underline{WettID},
WettKlasse, WettName, Geschlecht, Termin}

\begin{pspicture}(0,-1)(0,4)
\rput(11,2){ %
	\rput(-6,0){\SchuelerTraf}
	\rput(6,0){\Entity{Klasse}
		\rput(0,-1){\Attribut{KlName}{Klasse}}
		}
	\rput(0,0){\Relation{gehtIn}{gehtIn}{Schueler}{(1:1)}{Klasse}{(1:$\infty$)}
		}
}
\end{pspicture}

\Trafo{7cm}{Schüler}{\underline{SchID}, SchName, SchVorname, SchGeschlecht, SchGebDatum, {\color{red}KlName}}\hfill
\Trafo{3cm}{Klasse}{\underline{KlName}}

\subsection{Transformation von m:n Beziehungstypen}

\begin{pspicture}(0,-0.8)(0,2.5)
\rput(9.5,1.5){
\rput(-6,0){\Entity{Sportanlage} %
	\rput(-3,-1.5){\Key{AnlagenID}{Sportanlage}}
	\rput(1.5,-1.5){\Attribut{AnlagenName}{Sportanlage}}
	}
\rput(6,0){\Entity{Material}
	\rput(1.5,-1){\Attribut{MatAnzahl}{Material}}
	\rput(-0.8,-1.8){\Attribut{MatName}{Material}}
	\rput(-3,-1){\Key{MatID}{Material}}
	}
\rput(0,0){\Relation{benoetigt}{benötigt}{Sportanlage}{(1:$\infty$)}{Material}{(0:$\infty$)}
	\rput(-1,-1){\Attribut{Anzahl}{benoetigt}}
	}
}
\end{pspicture}

\Trafo{4.5cm}{Sportanlage}{\underline{AnlagenID}, AnlagenName}\hfill
\Trafo{4.7cm}{benoetigt}{{\color{red}\underline{AnlagenID}, \underline{MatID}}, Anzahl}\hfill
\Trafo{5.5cm}{Material}{\underline{MatID}, MatName, MatAnzahl}

\begin{pspicture}(0,-0.8)(0,3.8)
\rput(9.5,1.5){
\rput(-6,0){\Entity{Sportanlage} %
	\rput(-3,-1.5){\Key{AnlagenID}{Sportanlage}}
	\rput(1.5,-1.5){\Attribut{AnlagenName}{Sportanlage}}
	}
\rput(6,0){\WettkampfTraf}
\rput(0,0){\Relation{findetStatt}{findet statt}{Sportanlage}{(1:$\infty$)}{Wettkampf}{(0:3)}
	\rput(-1,-1.5){\Attribut{Dauer}{findetStatt}}
	}
}
\end{pspicture}

\Trafo{4.5cm}{Sportanlage}{\underline{AnlagenID}, AnlagenName}\hfill
\Trafo{4.5cm}{findet statt}{{\color{red}\underline{AnlagenID}, \underline{WettID}}, Dauer}\hfill
\Trafo{5.5cm}{Wettkampf}{\underline{WettID},
WettKlasse, WettName, Geschlecht, Termin}

\begin{pspicture}(0,-0.8)(0,4.5)
\rput(9.5,1.5){
\rput(-5,0){\SchuelerTraf}
\rput(6,0){\WettkampfTraf}
\rput(0.5,1){\Relation{nimmtTeil}{nimmt teil}{Schueler}{(3:5)}{Wettkampf}{(0:$\infty$)}
    \rput(-1,1.4){\Attribut{Platz}{nimmtTeil}}
    \rput(1.2,0.7){\Attribut{Ergebnis}{nimmtTeil}}
	}
\rput(0,-1){\Relation{verantw2}{verantw2}{Schueler}{(0:1)}{Wettkampf}{(1:$\infty$)}
		\rput(-0.5,-1){\Attribut{Funktion}{verantw2}}
		}
}
\end{pspicture}

\Trafo{5.3cm}{Schüler}{\underline{SchID}, SchName, SchVorname, SchGeschlecht, SchGebDatum, {\color{red}KlName}}\hfill
\Trafo{3.7cm}{nimmt teil}{{\color{red}\underline{SchID}, \underline{WettID}}, Platz,\\Ergebnis, {\color{red}Funktion}}\hfill
\Trafo{5cm}{Wettkampf}{\underline{WettID},
WettKlasse, WettName, Geschlecht, Termin}

Somit entfällt die Transformation der 1:n-Beziehung \textit{Schüler-verantw2-Wettkampf}. Das Attribut
 \textit{Funktion} wird zur Beziehung \textit{nimmt teil} übernommen.
\newpage
\section{Integritätsbedingungen}
\subsection{Statische Integritätsbedingungen}
Statische Bedingungen beziehen sich auf einen Datenbankzustand. Sie sind Einschränkungen und
 werden durch Prädikate bestimmt. In der Tabelle sind mögliche Wertebereiche angegeben. 
Die Einhaltung dieser Bedingungen wird an den Programmierer des Frontends übergeben.

\begin{tabular}[t]{lll}
  Attribut & Datentyp & Bedingung\\\hline\hline
\TabKopf{LEHRER}{3}
  LID & INTEGER(3) & \\
  LName & VARCHAR(32)	& \\
  LVorname & VARCHAR(32)	& \\
  WettID&INTEGER(3)&\\
  Funktion&VARCHAR(32)&Stationsleiter, Kampfrichter, Schreiber, Helfer\\
  KlName&VARCHAR(3)&\\\hline\hline
\TabKopf{KLASSE}{3}
  KlName&VARCHAR(3)&\\\hline\hline
\TabKopf{WETTKAMPF}{3}
  WettID&INTEGER(3)&\\
  WettName&VARCHAR(32)&z.B. Weitsprung\\
  Wettklasse&VARCHAR(32)&z.B. 7/8\\
  Geschlecht&CHAR&m, w\\
  Termin&TIMESTAMP&\\\hline\hline
  Attribut & Datentyp & Bedingung\\\hline\hline
\TabKopf{SCHUELER}{3}
  SchID&INTEGER(4)&\\
  SchName&VARCHAR(32)&\\
  SchVorname&VARCHAR(32)&\\
  SchGeschlecht&CHAR&m, w\\
  SchGebDatum&DATE&\\
  KlName&VARCHAR(3)&3\\\hline\hline
\TabKopf{SPORTANLAGE}{3}
  AnlagenID&INTEGER(2)&\\
  AnlagenName&VARCHAR(32)&\\\hline\hline
\TabKopf{MATERIAL}{3}
  MatID&INTERGER(3)&\\
  MatName&VARCHAR(32)&\\
  MatAnzahl&INTEGER(2)&\\\hline\hline
\end{tabular}\\
\newpage
\begin{tabular}[t]{lll}
\TabKopf{benötigt}{3}
  AnlagenID&INTEGER(2)&\\
  MatID&INTERGER(3)&\\
  Anzahl&INTEGER(3)&\\\hline\hline
\TabKopf{findet statt}{3}
  AnlagenID&INTEGER(2)&\\
  WettID&INTEGER(3)&max. 3 mal\\
  Dauer&INTEGER(3)&in Minuten\\\hline\hline
\TabKopf{nimmt teil}{3}
  SchID&INTEGER(4)&als Teilnehmer mind. 3, max. 5\\
  WettID&INTEGER(3)&\\
  Platz&INTEGER(3)&\\
  Ergebnis&INTEGER&Ergebnis in Meter, Sekunden, Anzahl, \dots \\
  Funktion&VARCHAR(32)&Stationsleiter, Kampfrichter, Schreiber, Helfer,Teilnehmer
\end{tabular}\\[3ex]

\subsection{Dynamische Integritätsbedingungen}
Dynamische Integritätsbedingungen beziehen sich auf Auswirkungen von Datenbankzustandsänderungen.\\
In unserer Miniwelt müssten beispielsweise alle Wettkampfteilnahmeeintäge gelöscht werden,
wenn ein Wettkampf gelöscht wird. 
Man muss auch beim Eintragen selbst darauf achten, dass ein Schüler nur einen
Wettkampf seiner Altersklasse und seines Geschlechtes absolviert.
Ähnliche Fragen könnte man bei allen Beziehungen diskutieren.
Das Setzen entsprechender Trigger würde den Unterrichtsrahmen sprengen 
und wird somit von uns nicht weiter verfolgt.

\newpage
\section{Funktionale Abhängigkeiten}

Für die Tabelle des Lehrers haben wir alle funktionalen Abhängigkeiten notiert. Bei der Tabelle 
Wettkampf haben wir die ein- und zweielementigen FAs notiert, die anderen aber erspart.
 
\begin{tabular}{r>{$}c<{$}l}
\TabKopf{LEHRER}{3}
\{LID\} &\rightarrow & \{LName, LVorname\}\\

\{LID, LName\} &\rightarrow & \{LVorname\}\\
\{LID, LVorname\} &\rightarrow &\{LName\}\\&&\\\hline

\TabKopf{KLASSE}{3}
\multicolumn{3}{l}{keine FAs}\\&&\\\hline

\TabKopf{WETTKAMPF}{3}

\multicolumn{3}{l}{WettID ist Schlüssel}\\

\{WettID\} &\rightarrow & \{WettName, Wettklasse, Geschlecht, Termin\}\\

\{WettID, WettName\} &\rightarrow & \{Wettklasse, Geschlecht, Termin\}\\
\{WettID, Wettklasse\} &\rightarrow & \{WettName, Geschlecht, Termin\}\\
\{WettID, Geschlecht\} &\rightarrow & \{WettName, Wettklasse, Termin\}\\
\{WettID, Termin\} &\rightarrow & \{WettName, Wettklasse, Geschlecht\}\\

\{WettID, WettName, Wettklasse\} &\rightarrow & \{Geschlecht, Termin\}\\
\{WettID, WettName, Geschlecht\} &\rightarrow & \{Wettklasse, Termin\}\\
\{WettID, WettName, Termin\} &\rightarrow & \{Wettklasse, Geschlecht\}\\

\{WettID, Wettklasse, WettName\} &\rightarrow & \{Geschlecht, Termin\}\\
\{WettID, Wettklasse, Geschlecht\} &\rightarrow & \{WettName, Termin\}\\
\{WettID, Wettklasse, Termin\} &\rightarrow & \{WettName, Geschlecht\}\\

\{WettID, Geschlecht, WettName\} &\rightarrow & \{Wettklasse, Termin\}\\
\{WettID, Geschlecht, Wettklasse\} &\rightarrow & \{WettName, Termin\}\\
\{WettID, Geschlecht, Termin\} &\rightarrow & \{WettName, Wettklasse\}\\

\{WettID, Termin,WettName\} &\rightarrow & \{Wettklasse, Geschlecht\}\\
\{WettID, Termin, Wettklasse\} &\rightarrow & \{WettName, Geschlecht\}\\
\{WettID, Termin, Geschlecht\} &\rightarrow & \{WettName, Wettklasse\}\\
\multicolumn{3}{l}{\dots jetzt mit 4 elementiger Mengen}\\

\TabKopf{SCHUELER}{3}
\multicolumn{3}{l}{SchID ist Schlüssel}\\
\{SchID\} &\rightarrow & \{SchName, SchVorname, SchGeschlecht, SchGebDatum\} \\
\TabKopf{ SPORTANLAGE}{3}
\multicolumn{3}{l}{AnlagenID ist Schlüssel}\\
\{AnlagenID\} &\rightarrow & \{AnlagenName\}\\
\{AnlagenName\} &\rightarrow & \{AnlagenID\}\\

\TabKopf{MATERIAL}{3}
\multicolumn{3}{l}{MatID ist Schlüssel}\\
\{MatID\} &\rightarrow & \{MatName, MatAnzahl\}\\
\end{tabular}

In \glqq benötigt\grqq, \glqq findet statt\grqq\: und \glqq nimmt teil\grqq\: sind keine funtionalen Abhängigeiten enthalten

\newpage
\section{Konkrete Umsetzung des Relationsentwurfes in einen Datenentwurf}
\subsection{Implementierung in Relationenalgebra (DES-RA)}
siehe Datenstruktur Ordner \glqq ralg\grqq
\subsection{Implementierung in SQL (postgreSQL)}
siehe Datenstruktur Ordner \glqq SQL\grqq
\newpage
\section{Beispielpopulationen}
\vspace*{-3ex}
\footnotesize\setlength{\tabcolsep}{2pt}
\begin{tabular}[t]{cllcll}
\TabKopf{LEHRER}{5}
LID&LName&LVorname&WettID&Funktion&KlName\\\hline
100&Bauer&Bernd		&100&Helfer	&7A\\
101&Becker&Petra	&101&Stationsleiter&00\\
102&Engel&Bernd		&102&Helfer	&8A\\
103&Fuchs&Sebastian	&103&Helfer	&11\\
104&Hofmann&Constanze	&104&Kampfrichter	&9A\\
105&Hofman&Conrad	&105&Helfer	&10A\\
106&Meier&Juergen	&106&Stationsleiter&12\\
107&Meier&Anja		&107&Stationsleiter&00\\
108&Schmidt&Anke	&100&Schreiber	&7A\\
109&Schulz&Klaus	&105&Stationsleiter&13
\end{tabular}\hfill
\begin{tabular}[t]{clccc}
\TabKopf{WETTKAMPF}{5}
 WettID&WettName&Wettklasse&Geschlecht&Termin\\\hline
100&Weitsprung&7/8&m&08:00\\
101&Weitsprung&7/8&w&08:20\\
102&100m&9/10&m&09:00\\
103&100m&9/10&w&09:00\\
104&Kugel&11/12&m&11:00\\
105&Kugel&11/12&w&11:15\\
106&Fußball&7/8&m&10:30\\
107&Fußball&7/8&w&10:30\\
108&Basketball&9/10&m&10:00\\
109&Basketball&9/10&w&10:00
\end{tabular}

\begin{tabular}[t]{cllclc}
\TabKopf{SCHUELER}{6}
 SchID&SchName&SchVorname&SchGeschlecht&SchGebDatum&KlName\\\hline
1000&Becker&Julian&m&2002-08-06&7A\\
1001&Braun&Monika&w&2002-12-05&7A\\
1002&Franke&Lina&w&2002-03-08&7A\\
1003&Kaiser&Lukas&m&2001-03-17&8A\\
1004&Merten&Paul&m&2001-06-22&8A\\
1005&Keller&Lukas&m&2001-03-18&8A\\
1006&Klein&Christian&m&2001-03-17&8A\\\hline
1007&Koch&Kristin&w&2000-03-17&9A\\
1008&Lange&Manuel&m&2000-12-22&9A\\
1009&Lorenz&Anton&m&2000-11-21&9A\\
1010&Meier&Julian&m&1999-01-01&10A\\
1011&Meier&Julian&m&1999-08-15&10A\\
1012&Meyer&Tom&m&1999-08-24&10A\\\hline
1013&Scholz&Werner&m&1998-05-06&11\\
1014&Weber&Chantal&w&1998-09-12&11\\
1015&Wolf&Simon&m&1998-07-15&11\\
1016&Zimmer&Erik&m&1997-06-23&12\\
1017&Fiebich&Simona&w&1997-05-25&12\\\hline
1018&Mutz&Gregor&m&1996-02-13&13\\
1019&Radig&Luise&w&1996-05-10&13\\
1020&Mader&Leonie&w&1996-07-20&13
\end{tabular}\hfill
\begin{tabular}[t]{r}
 \begin{tabular}[t]{clc}
\TabKopf{MATERIAL}{3}
 MatID&MatName&MatAnzahl\\\hline
100&Fußball&10\\
101&Basketball&10\\
102&Leibchen&30\\
103&5 kg Kugel&15\\
105&Maßband&20\\
106&Startklappen&10\\
107&Pfeifen&15\\
108&Schreibertische&20\\
109&Stühle&50\\
110&Stoppuhren&30
\end{tabular}\\
\begin{tabular}[b]{c}
\TabKopf{KLASSE}{1}
 KLName\\\hline
7A\\
8A\\
9A\\
10A\\
11\\
12\\
13\\
\end{tabular}
\begin{tabular}[b]{cl}
\TabKopf{SPORTANLAGE}{2}
 AnlagenID&AnlagenName\\\hline
10&Weitsprung 1\\
11&Weitsprung 2\\
12&Kugelstoß 1\\
13&Rasenplatz 1\\
14&Laufbahn 1\\
15&Laufbahn2\\
16&Ballspielplatz 1\\
17&Ballspielplatz 2\\
18&Turnhalle Feld 1
\end{tabular}
\end{tabular}

\begin{tabular}[t]{ccccl}
 \TabKopf{nimmt teil}{3}
  SchID&WettID&Platz&Ergebnis&Funktion\\\hline
1000&100&1&450&Teilnehmer\\
1000&106&1&0&Teilnehmer\\
1001&101&3&340&Teilnehmer\\
1001&107&2&0&Teilnehmer\\
1002&101&2&365&Teilnehmer\\
1002&107&1&0&Teilnehmer\\
1003&100&2&440&Teilnehmer\\
1003&106&2&0&Teilnehmer\\
1004&100&3&415&Teilnehmer\\
1004&106&2&0&Teilnehmer\\
1005&0&0&0&krank\\
1006&100&4&390&Teilnehmer\\
1007&103&2&130&Teilnehmer\\
1007&109&1&0&Teilnehmer\\
1008&0&0&0&krank\\
1009&102&1&122&Teilnehmer\\
1009&108&1&0&Teilnehmer\\
1010&102&2&130&Teilnehmer\\
1010&108&2&0&Teilnehmer\\
1011&102&3&132&Teilnehmer\\
1011&108&2&0&Teilnehmer\\
1012&102&4&133&Teilnehmer\\
1012&108&2&0&Teilnehmer\\
\end{tabular}
\hfill
\begin{tabular}[t]{l}
\begin{tabular}[t]{ccc}
 \TabKopf{findet statt}{3}
  AnlagenID&WettID&Dauer\\\hline
  10&100&20\\
  11&100&20\\
  10&101&20\\
  14&102&20\\
  15&103&20\\
  12&104&15\\
  12&105&15\\
  13&106&30\\
  16&107&20\\
  17&108&30\\
  16&109&20
\end{tabular}\hspace{1cm}
\begin{tabular}[t]{ccc}
 \TabKopf{benötigt}{3}
  AnlagenID&MatID&Anzahl\\\hline
  10&105&2\\
11&105&2\\
12&103&5\\
13&107&2\\
13&100&2\\
14&106&1\\
15&106&1\\
14&110&3\\
15&110&3\\
16&101&2\\
16&107&2\\
17&107&2\\
\end{tabular}\\
\begin{tabular}[b]{ccccl}
 \TabKopf{nimmt teil}{3}
  SchID&WettID&Platz&Ergebnis&Funktion\\\hline
1013&104&1&830&Teilnehmer\\
1014&105&1&655&Teilnehmer\\
1015&104&2&812&Teilnehmer\\
1016&0&0&0&krank\\
1017&105&1&655&Teilnehmer\\
1018&100&0&0&Helfer\\
1019&101&0&0&Helfer\\
1020&107&0&0&Helfer\\
\end{tabular}
\end{tabular}
\newpage
\normalsize
\section{Systematischer Test der Datenbank}
Die Datenbank haben wir systematisch mit einfachen DML-Konstrukten getestet. Die Testdatei liegt im Ordner \glqq SQL\grqq unter dem Namen \glqq DML-in-SQL.sql\grqq
.
\section{Aufgabensequenz von einfachen bis sehr schwierigen Anfragen mit Lösungen}
Die folgenden Abfragen haben wir in DES-RA und in postgreSQL realisiert.
\begin{enumerate}
 \item Welcher Lehrer bereut die Klasse 7A?
 
 \item Welcher Lehrer ist ein Stationsleiter?

 \item Welche Klasse gewann das Basketballturnier der Jungen?

 \item Nenne alle weiblichen Schülerinnen.
 \item Ordne alle männlichen Schüler nach Klassen.
 \item Wie viele Goldmedaillen werden benötigt?
 \item Wie viele Krankmeldungen gibt es?
 \item Wie viele Schüler sind zum 100m-Lauf gemeldet?
 \item Wie viele Schüler nehmen am 100-m-Lauf teil?
 \item Welche Materialien werden benötigt?
 \item Wie viele Pfeifen werden benötigt?
 \item Wie viele der jeweiligen Materialien werden benötigt?
 \item Welcher Lehrer leitet den Wettkampf Weitsprung 7/8 w?
 \item Welche Disziplinen wurden am Sportfest angeboten?
 \item An welchem Wettkämpfen nimmt Julian Becker teil?
 \item Welcher Schüler hilft beim Wettkampf Weitsprung 7/8 w?
 \item Welche Sportanlagen werden benötigt?
 \item Welche Sportanlagen werden nicht benötigt?
 \item Auf welchen Sportanlagen beginnt um 09:00 ein Wettkampf?
 \item Welche Wettkämpfe haben die Schüler der 10A zu welcher Zeit und wo?
 \item Welche Klassen haben 1. Plätze bei Wettkämpfen?
 \item Wie viele Schüler waren beim Sportfest als Helfer tätig?
 \item Wie viel Meter sprang der beste Schüler ?
 \item  Welche Materialien muss ein Lehrer besorgen?
 \item An welchen Wettkämpfen, wo und wann, nehmen alle Schüler mit dem Vor-
namen Julian teil?
 \item Welche Sportanlage ist um 9:00 Uhr frei?
 \item Welche Sportanlagen sind um 10 Uhr noch frei/nutzbar?
 \item Welche Klassen belegten im Fußball den 2. Platz?
 \item Welche Schüler nehmen nur an einem Wettkampf teil und sind nicht krank?
 \item Wie war die Reihenfolge beim 100m Lauf der Jungen aus Klasse 9/10?
 \item Gib eine nach Wettkämpfen und Platzierungen sortierte Tabelle
aller Schüler heraus.
 \item  Welche Schüler liefen die 100m zwischen 130 und 132 Zehntelsekunden?
 \item Auf  welchen Anlagen findet der 100 m Lauf statt
 \item Wer ist bei einem Wettkampf unter einem falschen Geschlecht eingetragen?
 \item Welche nicht als Helfer eingesetzten Lehrer müssen zu welcher Zeit bei den 
Wettkämpfen sein und welche Funktion haben sie an der jeweiligen 
Sportanlage?

\end{enumerate}

\subsection{Relationenalgebra (DES-RA)}
Die umgesetzten Anfragen und Anfrageergebnisse findet man im Ordner \glqq ralg\grqq.
\subsection{SQL (postgreSQL)}
Die umgesetzten Anfragen und Anfrageergebnisse findet man im Ordner \glqq sql\grqq.
\section{Implementierung einer Sicht webbasiert in PHP}
\subsection{Begründung und Motivation der Sicht}
Aus Schülersicht erschien es uns attraktiv, die Möglichkeit des Internets zu nutzen und eine webbasierte Abfrage zu gestalten.
Um webbasiert Abfragen zu gestalten, wird in vielen Fällen PHP benutzt. PHP ist schnell erlernbar, ist kostenlos und besitzt 
Weiterentwicklungsmöglichkeiten und Differenzierungspotential.\\

Eine schülerrelevante Anfrage an die Datenbank wäre die nach den zu absolvierenden Wettkämpfen inklusive Zeiten und Orten.
Diese Anfrage bietet eine Wiederholung von SQL-Abfragen und deren übersichtlichen Darstellung auf der WEB-Seite.
\subsection{Beschreibung der interaktiven Basismöglichkeiten}
Folgende Suchmöglichkeiten enthält unsere Abfrageseite
\begin{itemize}
 \item Eingabe des Vornamens
\item Eingabe des Nachnamens,
\item Eingabe von Vor- und Nachnamen
\item keine Eingabe
\end{itemize}
Als Ergebnis wird die jeweilige Antwort auf die Suchanfrage tabellarisch ausgegeben, wobei keine Eingabe die Ausgabe aller Schüler
mit ihren Wettkämpfen nach sich zieht.\\

Das Beispiel befindet sich im Ordner \glqq view\grqq .
\subsection{Optionale Möglichkeiten}
Als Weiterführung könnte die Übersichtlichkeit der WEB-Seite mit einem Menü erweitert werden, welches beispielsweise
Datenbankeingaben und weitere Abfragen per Knopfdruck zulässt.
\newpage
\section{Didaktische und methodische Überlegungen}
Das Thema \glqq Datenbanken\grqq ist ein wesentlicher Bestandteil des Rahmenlehrplanes Informatik der gymnasialen Oberstufe in Berlin.
Darin steht: \glqq Am Beispiel der Entwicklung eines Datenbaksystems führen die Schüler alle Phasen des Problemlöseprozesses 
von der Analyse der Ausgangssituation der Daten über die Modellierung einer Datenbank bis hin zu ihrer praktischen Umsetzung in einem
 Datenbankmanagementsystem selbstständig durch.\grqq

Darüber hinaus werden im Grund- und Leistungskurs (in-1, IN-1) Datenbanken mit Softwareentwicklung, wie z.B. PHP, verknüpft. 

Unsere Miniwelt halten wir als exemplarisches Beispiel geeignet. Insbesondere bietet sie gute Diffenzierungsmöglichkeiten hinsichtlich 
Verkleinerung oder Vergößerung des gewählten Miniweltausschnittes.\\

Unsere gewählte Miniwelt halten wir vom Schwierigkeitsgrad, Umfang und Zeitbedarf realisierbar im Grund- und Leistungskurs.
Anhand dieses Beispieles können auch theoretische Aspekte der Datenbankentwicklung vermitelt werden und so der komplette Zeitumfang 
eines Semesters genutzt werden. \\
Möglich wäre auch die Vergabe als Projekt nach theoretischer Einführung in das Thema.

Im Grundkurs würden wir DES-RA nicht berücksichtigen und einen größeren Fokus auf die webbasierte Programmierung in PHP legen, jedoch empfinden wir SQL
 als unverzichtbaren Bestandteil des Themas Datenbanken.

Ausbaupotential bietet vor Allem die PHP-Anbindung. Hier wäre z.B. eine umfängliche WEB-Seite mit Möglichkeiten der Datenmanipulation
und ergänzender Abfragen denkbar.
\section{Arbeitsverteilung}
\begin{tabular}{ll}
Teile des Datenbankprojektes & Bearbeiter\\\hline
Beschreibung der Miniwelt&Huth, Kreißig, Petri\\
ER-Modell&Huth, Kreißig, Petri\\
Transformation&Kreißig\\
Integritätsbedingungen&Petri\\
Funktionale Abhängigkeiten&Huth\\
Implementierung in DES-RA und postgreSQL&Huth, Kreißig, Petri\\
Beispielpopulationen&Kreißig, Petri\\
Systematischer Test mit DML&Huth, Kreißig\\
Aufgabensequenzen in postgreSQL&Huth, Kreißig, Petri\\
Adaption an DES-RA&Kreißig\\
Implementierung in PHP&Huth, Kreißig, Petri\\
Didaktisch-methodische Überlegungen&Huth, Kreißig, Petri\\
Bereitstellung in \LaTeX &Kreißig
\end{tabular}

\end{document}
